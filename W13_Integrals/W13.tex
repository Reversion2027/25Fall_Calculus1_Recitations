\documentclass[]{beamer}
\usepackage[utf8]{inputenc}
\usepackage{xeCJK}
\usepackage{graphicx}
\usepackage{subfigure}
\usepackage{mathtools}
\usepackage{utopia} %font utopia imported
\usetheme{CambridgeUS}
\usecolortheme{dolphin}
\usefonttheme{professionalfonts}
\usepackage{natbib}
\usepackage{hyperref}
\usepackage{fontspec}
\usepackage{setspace}
\usepackage{float}
\usepackage{extarrows}
% \usepackage{enumitem}

\setCJKmainfont{SourceHanSansSC-Regular.otf}[Path=../, BoldFont=bold.otf]

\setbeamerfont{title}{size=\Large}
\setbeamerfont{subtitle}{size=\small}
\setbeamerfont{date}{size=\small}
\setbeamerfont{institute}{size=\small}

\setstretch{1.3}
% \setlength{\parindent}{2em}
% \setlength{\parskip}{0pt}

% \setlist[itemize]{leftmargin=2em}

% ↓↓↓ Modify this ↓↓↓
\title{高等数学I\quad 习题课13}
\subtitle{不定积分与定积分}
\date[2025.12.11]{2025.12.11}
% ↑↑↑ Modify this ↑↑↑

\author[上海科技大学]{}
\institute[]{上海科技大学}

\begin{document}

\begin{frame}
    \vspace{15pt}
    \titlepage
\end{frame}

\begin{frame}{习题课12 反馈}
    \begin{itemize}
        \item 后续章节解题技巧性提高,希望能涉及更多方法、技巧
        \begin{itemize}
            \item 会进行相关讲解,尤其是积分的多种计算方法
        \end{itemize}
    \end{itemize}
\end{frame}


\begin{frame}{目录}
    \vfill
    \tableofcontents[hideallsubsections]
    \vfill
\end{frame}

\AtBeginSection[ ]
{
\begin{frame}{目录}
    \vfill
    \tableofcontents[currentsection,hideallsubsections]
    \vfill
\end{frame}
}

\section{不定积分}

\begin{frame}{回顾}
    \begin{align*}
        \int f[\varphi(x)]\varphi'(x)\mathrm dx&=F[\varphi(x)]+C\\
        \int f(x)\mathrm dx &= F[\varphi^{-1}(x)]+C\\
        \int u(x)v'(x)\mathrm dx&=u(x)v(x) - \int u'(x)v(x)\mathrm dx
    \end{align*}
\end{frame}

\begin{frame}{例}
    求
    \[
    \int \frac{\mathrm dx}{e^x-e^{-x}}
    \]
\end{frame}

\begin{frame}{例}
    求
    \[
    \int \frac{x\tan \sqrt{1+x^2}}{\sqrt{1+x^2}}\mathrm dx
    \]
\end{frame}

\begin{frame}{例(重访,教材 例5.39)}
    求
    \[
    \int e^{ax}\sin bx\mathrm dx \quad (a\ne 0)
    \]
\end{frame}

\begin{frame}{有理函数}
    \textbf{有理函数}是指两个实系数多项式的商,一般形式为
    \[
    R(x)=\frac{P_n(x)}{Q_m(x)}
    \]
    其中$P,Q$分别是$n,m$次实系数多项式.

    由定义知:$n$与$m$可以相等也可以不等;虽然称为“有理”函数,但多项式的系数可以是无理数.

    当$n<m$,称$R(x)$是真分式,否则为假分式.
\end{frame}

\begin{frame}{代数学基本定理}
    一个真分式$R(x)$必定能分解为以下四种分式之和:
    \[
    \frac{A}{x-a},\frac{A}{(x-a)^k},\frac{Bx+D}{x^2+px+q},\frac{Bx+D}{(x^2+px+q)^k}
    \]
    则可以逐项进行积分. 前两者的积分是显然的;后两者的积分需要多加练习,学会自行推导.

    本定理要求熟练掌握,但不要求理解其原理.
\end{frame}

\begin{frame}{例5.44}
    求
    \[
    \int\frac{\mathrm dx}{x(x^{10}+1)}
    \]
\end{frame}

\begin{frame}{注意}
    \begin{itemize}
        \item 使用有理函数积分做部分分式展开时,分母的次数不宜过高,否则拆得的项太多、过于复杂,此时应使用拆项法
    \end{itemize}
\end{frame}

\begin{frame}{例}
    求
    \[
    \int \frac{x^2+2}{(x+1)^4}\mathrm dx
    \]
\end{frame}

\begin{frame}{三角函数有理式的积分}
    \textbf{三角函数有理式}指形如$R(\sin x,\cos x)$的函数,其中$R(u,v)$是将$(u,v)$经有理运算所得的表达式. 可分为以下类型:
    \begin{itemize}
        \item $R(\sin x)\cos x\mathrm dx$或$R(\cos x)\sin x\mathrm dx$
        \item $R(\sin^2x, \cos^2x)\mathrm dx$
        \item $\cos mx\cos nx\mathrm dx, \sin mx\sin nx\mathrm dx, \ldots$
    \end{itemize}
\end{frame}

\begin{frame}{三角函数有理式的积分}
    \begin{itemize}
        \item $R(\sin x)\cos x\mathrm dx$或$R(\cos x)\sin x\mathrm dx$:转化为$R(\sin x)\mathrm d\sin x,R(\cos x)\mathrm d\cos x$
        \item $R(\sin^2x, \cos^2x)\mathrm dx$:作代换$\tan x = u$
        \item $\cos mx\cos nx\mathrm dx, \sin mx\sin nx\mathrm dx, \ldots$:和差化积(学会推导!)
    \end{itemize}
\end{frame}

\begin{frame}{万能代换}
    令$t=\tan (x/2)$,则
    \[
    \sin x = \frac{2t}{1+t^2},\cos x = \frac{1-t^2}{1+t^2},\mathrm dx=\frac{2\mathrm dt}{1+t^2}
    \]
    将关于三角函数的有理函数转化为关于单变量$t$的有理函数,便于积分.
\end{frame}

\begin{frame}{简单无理函数}
    若被积函数含有根式$\sqrt[n]{ax+b},\displaystyle\sqrt[n]{\frac{ax+b}{cx+d}}$,常采取第二换元法化为有理函数的积分.
\end{frame}

\begin{frame}{例5.50}
    求
    \[
    \int \frac1x\sqrt{\frac{x+1}{x}}\mathrm dx
    \]
\end{frame}

\begin{frame}{例5.51}
    求
    \[
    \int \frac{\mathrm dx}{\sqrt{\sqrt{x}-1}}
    \]
\end{frame}

\begin{frame}{21Fall, Final Exam, 13}
    求
    \[
    \int \frac{\sqrt{x^2-4}}{x}\mathrm dx
    \]
\end{frame}

\section{定积分}

\subsection{定义}

\begin{frame}{定义}
    若函数$f$在区间$[a,b]$上由定义且有界,若$\exists I\in \mathbb R,\forall \varepsilon > 0,\exists\delta > 0$,使得
    对任意$[a,b]$的划分(\textit{partition})
    \[
    a=x_0<x_1<x_2<\cdots<x_n=b,
    \]
    及任意$\xi_i\in [x_{i-1},x_i], 1\le i\le n$,有1
    \[
    \lambda=\max\{x_i-x_{i-1}\}<\delta\quad \Rightarrow \quad |\sum_{i=1}^{n}f(\xi_i)(x_i-x_{i-1})-I|<\varepsilon
    \]
    则称$f$在$[a,b]$上\textbf{黎曼可积},记$f\in R[a,b]$. 极限$I$叫做$f$在$[a,b]$上的定积分,记为$\displaystyle I=\int_a^bf(x)\mathrm dx$.
\end{frame}

\begin{frame}{理解}
    \begin{itemize}
        \item 定义中,划分与$\xi_i$的选取都是任意的
        \item 几何意义:曲线与$x$轴围成的区域的(有符号)面积
        \item 函数可以有有限个间断点:分段进行积分.
        \item 充分条件:$f$在$[a,b]$上满足以下条件之一:
        \begin{itemize}
            \item 连续
            \item 有界且有有限个间断点
            \item 单调
        \end{itemize}
    \end{itemize}
\end{frame}

\begin{frame}{例5.2}
    使用定义计算定积分
    \[
    \int_0^1 x^2\mathrm dx
    \]    
\end{frame}

\begin{frame}{例5.3}
    使用定义计算定积分
    \[
    \int_1^2 \frac{1}{x}\mathrm dx
    \]    
\end{frame}

\begin{frame}{例5.4}
    已知$\displaystyle \int_0^1x^p\mathrm dx=\frac1{p+1}$,计算极限
    \[
    \lim_{n\rightarrow \infty}\frac{1^p+2^p+\cdots+n^p}{n^{p+1}}
    \]
\end{frame}

\subsection{性质}

\begin{frame}{定积分的性质}
    \begin{itemize}
        \item 上下限交换
        \[
        \int_a^b f(x)\mathrm dx=-\int_b^af(x)\mathrm dx,\int_a^af(x)\mathrm dx=0
        \]
    \end{itemize}
\end{frame}

\begin{frame}{定积分的性质}
    \begin{itemize}
        \item 线性性:
        \[
        \int_a^b[\alpha f(x)+\beta g(x)]\mathrm dx=\alpha \int_a^b f(x)\mathrm dx+\beta\int_a^bg(x)\mathrm dx
        \]
        \item 区间可加性:设函数$f\in R[a,b],c\in(a,b)$,则$f\in R[a,c],f\in R[c,b]$且
        \[
        \int_a^bf(x)\mathrm dx=\int_a^cf(x)\mathrm dx+\int_c^bf(x)\mathrm dx
        \]
    \end{itemize}
\end{frame}

\begin{frame}{定积分的性质}
    \begin{itemize}
        \item 保号性:设函数$f\in R[a,b]$,且在$[a,b]$上$f(x)\ge 0$,则
        \[
        \int_a^b f(x)\mathrm dx\ge 0.
        \]
    \end{itemize}
\end{frame}

\begin{frame}{保号性 推论}
    \begin{itemize}
        \item 保序性:设函数$f,g\in R[a,b]$,且在$[a,b]$上$f(x)\le g(x)$,则
        \[
        \int_a^b f(x)\mathrm dx\le \int_a^b g(x)\mathrm dx
        \]
        \item 估值不等式:设函数$f,g\in R[a,b]$,且在$[a,b]$上$m\le f(x)\le M$,则
        \[
        m(b-a)\le \int_a^b f(x)\mathrm dx\le M(b-a)
        \]
        \item 绝对值不等式:设函数$f\in R[a,b]$,且$|f|\in R[a,b]$,则
        \[
        \left|\int_a^bf(x)\mathrm dx\right|\le \int_a^b|f(x)|\mathrm dx
        \]
    \end{itemize}
\end{frame}

\begin{frame}{例5.5}
    估计定积分
    \[
    \int_{\frac12}^1\sqrt{1-x^2+x^3}\mathrm dx
    \]
    的值.    
\end{frame}

\begin{frame}{例5.6}
    设函数$f(x)\in C[a,b]$,且在$[a,b]$上,$f(x)\ge 0$. 若$\displaystyle\int_a^b f(x)=0$,求证
    \[
    f(x)\equiv 0.
    \]
\end{frame}

\begin{frame}{定积分的性质(Schwarz不等式)}
    设函数$f(x),g(x)\in C[a,b]$,则
    \[
    \left(\int_a^b f(x)g(x)\mathrm dx\right)^2\le \int_a^bf^2(x)\mathrm dx\int_a^bg^2(x)\mathrm dx
    \]
\end{frame}

\subsection{积分中值定理}

\begin{frame}{积分中值定理}
    设函数$f\in C[a,b],g\in R[a,b]$,且$g(x)$在$[a,b]$上不变号,则
    \[
    \exists \xi\in[a,b],\ \text{s.t.}\ \int_a^b f(x)g(x)\mathrm dx=f(\xi)\int_a^bg(x)\mathrm dx
    \]
\end{frame}

\begin{frame}{例5.8}
    求极限
    \[
    \lim_{n\rightarrow\infty}\int_0^{\frac12}\frac{x^n}{\sqrt{1+x^2}}\mathrm dx
    \]
\end{frame}

\begin{frame}{例5.9}
   设函数$f\in C[0,1]$,且$f\in D(0,1)$,又$f(1)=2\int_0^{\frac12}xf(x)\mathrm dx$,证明:
   \[
   \exists \xi\in(0,1),\ \text{s.t.}\ f(\xi)+\xi f'(\xi)=0.
   \] 
\end{frame}

% -------------------------------------------------------
% \section*{}
% \begin{frame}{反馈问卷}
%     \begin{figure}[H]
%         \centering
%         \includegraphics[width=0.5\linewidth]{qrcode.png}
%     \end{figure}
% \end{frame}
% \begin{frame}
% \vspace{25pt}
% \[
% \text{\Huge }
% \]
% \end{frame}

\end{document}
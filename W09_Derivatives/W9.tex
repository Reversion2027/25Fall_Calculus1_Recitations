\documentclass[]{beamer}
\usepackage[utf8]{inputenc}
\usepackage{xeCJK}
\usepackage{graphicx}
\usepackage{subfigure}
\usepackage{mathtools}
\usepackage{utopia} %font utopia imported
\usetheme{CambridgeUS}
\usecolortheme{dolphin}
\usefonttheme{professionalfonts}
\usepackage{natbib}
\usepackage{hyperref}
\usepackage{fontspec}
\usepackage{setspace}
\usepackage{float}
\usepackage{extarrows}
% \usepackage{enumitem}

\setCJKmainfont{SourceHanSansSC-Regular.otf}[Path=../fonts/, BoldFont=bold.otf]

\setbeamerfont{title}{size=\Large}
\setbeamerfont{subtitle}{size=\small}
\setbeamerfont{date}{size=\small}
\setbeamerfont{institute}{size=\small}

\setstretch{1.3}
% \setlength{\parindent}{2em}
% \setlength{\parskip}{0pt}

% \setlist[itemize]{leftmargin=2em}

% ↓↓↓ Modify this ↓↓↓
\title{高等数学I\quad 习题课09}
\subtitle{导数,期中复习}
\date[2025.11.13]{2025.11.13}
% ↑↑↑ Modify this ↑↑↑

\author[上海科技大学]{}
\institute[]{上海科技大学}

\begin{document}

\begin{frame}
    \vspace{15pt}
    \titlepage
\end{frame}

% \begin{frame}{Quiz}
%     \[
%     \text{\Huge 18:00 - 18:40}
%     \]
% \end{frame}

\begin{frame}{习题课08 反馈}
    \begin{columns}
        % 左栏:文字
        \begin{column}{0.5\textwidth}
            \begin{figure}[H]
                \centering
                \includegraphics[width=1.0\linewidth]{fb.png}
                \caption{课程质量}
            \end{figure}
        \end{column}

        \begin{column}{0.5\textwidth}
            \begin{figure}[H]
                \centering
                \includegraphics[width=1.0\linewidth]{fb.png}
                \caption{课堂氛围}
            \end{figure}
        \end{column}
    \end{columns}
\end{frame}

% \begin{frame}{Before we start\dots}
%     \begin{itemize}
%         \item 模拟卷1、2已发放,请根据自身需求,有选择性地完成.
%         \item 关于是否要追加讲评课的调查:
%     \end{itemize}
%     \begin{figure}[H]
%         \centering
%         \includegraphics[width=0.4\linewidth]{pollforexam.png}
%     \end{figure}
% \end{frame}

% \begin{frame}{补充材料}
%     \url{https://www.bilibili.com/video/BV1qW411N7FU}
%     \begin{figure}[H]
%         \centering
%         \includegraphics[width=0.8\linewidth]{3b1b.jpg}
%     \end{figure}
% \end{frame}

\begin{frame}{目录}
    \tableofcontents
\end{frame}

\AtBeginSection[ ]
{
\begin{frame}{目录}
    \tableofcontents[currentsection]
\end{frame}
}

\section{微分}

\begin{frame}{思考}
    考虑如何数值拟合$e^{0.01}$的值,误差不超过$0.01$.
\end{frame}

\begin{frame}{拟合方法}
    \begin{itemize}
        \item 导数:函数在某点附近的最佳线性拟合
        \item \textbf{函数可导}:在足够小的范围之内,函数可以近似地被视为直线\\
        $\Rightarrow$使用导数计算出函数的近似值,进行\textbf{非线性函数的局部线性化}
    \end{itemize}
\end{frame}

\begin{frame}{微分}
    例3.13
    \begin{itemize}
        \item 一个半径为$r$的金属球,因温度改变,半径变为$r+\Delta r$,求体积的改变量
        \begin{align*}
        \Delta V&=\frac43\pi[(r+\Delta r)^3-r^3]=4\pi r^2\Delta r+4\pi r(\Delta r)^2+\frac43\pi(\Delta r)^3\\
        &=4\pi r^2\Delta r+o(\Delta r)
        \end{align*}
        \item $\Delta V$ 由关于$\Delta r$的线性函数与$\Delta r$的高阶无穷小(因为$|\Delta r|$很小)组成.
        后者在$\Delta r$足够小时可以忽略不计
    \end{itemize}
\end{frame}

\begin{frame}{微分}
    一般地,若函数$y=f(x)$可导,且在点$x_0$附近,自变量发生了一段微小改变$\Delta x$, 则因变量$y$随之发生的改变满足关系
    \[
    \Delta y=f'(x_0)\Delta x + o(\Delta x)
    \]
    当$\Delta x\rightarrow 0$,$o(\Delta x)$可忽略不计,此时我们可以使用$f'(x_0)\Delta x$来估计$\Delta y$的值,且误差不大于$\Delta x$.

    (如何进一步增加精度?)
\end{frame}

\begin{frame}{一阶微分形式不变性}
    对于可微函数$y=f(x)$,记其微分为$\mathrm dy=f'(x)\mathrm dx$. 将$x$换成任一可微函数$\varphi(x)$,公式仍成立:
    \[
    \mathrm dy = f'(\varphi(x))\mathrm d\varphi(x)
    \]
\end{frame}

\section{链式法则与反函数求导}

\begin{frame}{导数的多种记号}
    \setlength{\leftmargini}{7em}
    \begin{itemize}
        \item [Leibniz] $\displaystyle\frac{\mathrm dy}{\mathrm dx}\qquad$ (Recommended)
        \item [Lagrange] $f'(x)$
        \item [Newton] $\dot{y}$
    \end{itemize}
\end{frame}

\begin{frame}{链式法则}
    设函数$u=\varphi(x)$在$x$处可导,函数$y=f(u)$在对应$x$的点$u$处可导,则复合函数$y=f[\varphi(x)]$在点$x$处可导,且
    \[
    \frac{\mathrm dy}{\mathrm dx}=(\frac{\mathrm d}{\mathrm du}y)\cdot(\frac{\mathrm d}{\mathrm dx}u)
    \]
    或
    \[
    \frac{\mathrm dy}{\mathrm dx}=f'[\varphi(x)]\varphi'(x)
    \]
\end{frame}

\begin{frame}{链式法则}
    求函数
    \[
    f(x)=\sin(x^2)
    \]
    的导函数.
\end{frame}

\begin{frame}{对数求导法}
    求函数
    \[
    y=\frac{e^{2x}\sin^4x}{\sqrt[3]{2x-1}(4x+3)^2}
    \]
    的导函数.

    $\Rightarrow$对数求导法不仅可以用于指数的求导,也可以用于复杂因式的求导
\end{frame}

\begin{frame}{反函数求导}
    考虑两个函数的图像关系:
    \[
    y=x^2(x>0),y=\sqrt x
    \]
    互为反函数的两个函数,图像关于$y=x$对称.
    
    $\Rightarrow$对应点的切线斜率互为倒数
\end{frame}

\begin{frame}{反函数求导}
    函数$x=f(y)$时区间$I$上的严格单调可导函数,且$f'(y)\ne 0$,则它的反函数$y=f^{-1}(x)$在对应点$x$处可导,且
    \[
    \displaystyle\frac{\mathrm dy}{\mathrm dx}=\frac{1}{\displaystyle \frac{\mathrm dx}{\mathrm dy}}.
    \]
    证明:考虑复合函数$f^{-1}(f(x))$的导函数.
\end{frame}

\begin{frame}{例}
    求函数$y=\arcsin x$的导数.
\end{frame}

\section{隐函数与参数方程求导}

\begin{frame}{从相关变化率开始\dots}
    一把$5$米长的梯子直靠在墙壁上,底端距离墙面$0$米.现在梯子在墙上的一端以$1$m/s的速度下滑,求
    梯子底端速度随时间变化的关系式. (不允许使用速度关联)
\end{frame}

\begin{frame}{思路}
    存在方程$x(t)^2+y(t)^2=5$.
    \begin{enumerate}
        \item 既然$x>0,y>0$,不妨记$x(t)=\sqrt{5-y(t)^2}$,再使用链式法则
        \item 对等式两边求导:
        \[
        \frac{\mathrm d}{\mathrm dt}(x(t)^2+y(t)^2) = 0
        \]
        "当$x$和$y$分别都变动一点的时候,函数值会变动多少?"$\Rightarrow 0.$

        那么,$\Delta x$与$\Delta y$必然满足如下关系\dots
        \[
        2x\Delta x+2y\Delta y = 0.
        \]
    \end{enumerate}
\end{frame}

\begin{frame}{隐函数求导}
    \begin{itemize}
        \item 隐函数表征了多个变量之间的固定关系.不论这个关系如何改变,隐函数的等式始终成立.\\
        $\Rightarrow$将隐函数化为$f(x,y)=0$的形式,则左边函数的变化率始终为$0$.
        \item 对$f(x,y)$求微分,即可得到$\mathrm dx,\mathrm dy$之间的关系
        \item (这一部分的知识在学习了第8章-多元微分学后会有更深刻的理解\dots 现在先接受它!)
    \end{itemize}
\end{frame}

\begin{frame}{参数方程求导}
    对于参数方程决定的函数
    \[
    \left\{\begin{array}{l}
        x=\varphi(t)\\
        y=\psi(t)
    \end{array},\right.
    \]
    若$\varphi(t),\psi(t)$可导,则
    \[
    \frac{\mathrm dy}{\mathrm dx}=(\frac{\mathrm d}{\mathrm dt}y)\Big/(\frac{\mathrm d}{\mathrm dt}x)
    \]
\end{frame}

\begin{frame}{又一次回到相关变化率}
    一把$5$米长的梯子直靠在墙壁上,底端距离墙面$0$米.现在梯子在墙上的一端以$1$m/s的速度下滑,求
    梯子底端速度随时间变化的关系式.
\end{frame}

\begin{frame}{参数方程求导}
    再一次考虑
    \[
    \left\{\begin{array}{l}
        x=\varphi(t)\\
        y=\xi(t)
    \end{array},\right.
    \]
    “当$t$变动一点的时候,$x$和$y$分别会变动多少?”
\end{frame}



\section{高阶导数}



\begin{frame}{定理}
    设函数$u(x),v(x)$在区间$I$上$n$阶可导,$\alpha,\beta\in\mathbb R$,则:
    \begin{itemize}
        \item $\alpha u(x)+\beta v(x), u(x)v(x)$在$I$上均$n$阶可导
        \item $[\alpha u(x)+\beta v(x)]^{(n)}=\alpha u^{(n)}(x)+\beta v^{(n)}(x)$
        \item $\displaystyle[u(x)v(x)]^{(n)}=\sum_{k=0}^n{n\choose k}u^{(k)}(x)v^{(n-k)}(x)$
    \end{itemize}
    证明:数学归纳法
\end{frame}

% \section*{}
% \begin{frame}{反馈问卷}
%     \begin{figure}[H]
%         \centering
%         \includegraphics[width=0.5\linewidth]{qrcode.png}
%     \end{figure}
% \end{frame}


% -------------------------------------------------------
% \section*{}
% \begin{frame}
% \vspace{25pt}
% \[
% \text{\Huge Office Hour}
% \]
% \end{frame}

\end{document}
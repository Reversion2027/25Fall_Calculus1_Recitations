\documentclass[]{beamer}
\usepackage[utf8]{inputenc}
\usepackage{xeCJK}
\usepackage{graphicx}
\usepackage{subfigure}
\usepackage{mathtools}
\usepackage{utopia} %font utopia imported
\usetheme{CambridgeUS}
\usecolortheme{dolphin}
\usefonttheme{professionalfonts}
\usepackage{natbib}
\usepackage{hyperref}
\usepackage{fontspec}
\usepackage{setspace}
\usepackage{float}
\usepackage{extarrows}
% \usepackage{enumitem}

\setCJKmainfont{SourceHanSansSC-Regular.otf}[Path=../fonts/, BoldFont=bold.otf]

\setbeamerfont{title}{size=\Large}
\setbeamerfont{subtitle}{size=\small}
\setbeamerfont{date}{size=\small}
\setbeamerfont{institute}{size=\small}

\setstretch{1.3}
% \setlength{\parindent}{2em}
% \setlength{\parskip}{0pt}

% \setlist[itemize]{leftmargin=2em}

% ↓↓↓ Modify this ↓↓↓
\title{高等数学I\quad 习题课11}
\subtitle{微分中值定理}
\date[2025.11.27]{2025.11.27}
% ↑↑↑ Modify this ↑↑↑

\author[上海科技大学]{}
\institute[]{上海科技大学}

\begin{document}

\begin{frame}
    \vspace{15pt}
    \titlepage
\end{frame}

% \begin{frame}{习题课09 反馈}
%     \begin{columns}
%         % 左栏:文字
%         \begin{column}{0.5\textwidth}
%             \begin{figure}[H]
%                 \centering
%                 \includegraphics[width=1.0\linewidth]{fb1.png}
%                 \caption{课程质量}
%             \end{figure}
%         \end{column}

%         \begin{column}{0.5\textwidth}
%             \begin{figure}[H]
%                 \centering
%                 \includegraphics[width=1.0\linewidth]{fb2.png}
%                 \caption{课堂氛围}
%             \end{figure}
%         \end{column}
%     \end{columns}
% \end{frame}

% \begin{frame}{后续安排(暂定)}
%     \begin{itemize}
%         \item W11 微分中值定理
%         \item W12 微分中值定理\ 函数性态的研究
%         \item W13 定积分及其性质\ 积分中值定理
%         \item W14 不定积分\ 定积分的计算
%         \item W15 反常积分\ 级数
%         \item W16 级数
%         \item (模拟卷讲评+期末复习?)
%     \end{itemize}
% \end{frame}

% \begin{frame}{作业提交}
%     如果match错误,助教需要在如下图片中找到你的作业:
%     \begin{figure}[H]
%         \centering
%         \includegraphics[width=1.0\linewidth]{hwhandin.png}
%     \end{figure}
%     Gradescope似乎不会把逐张提交的png图片按照题目顺序排序\dots
% \end{frame}


\begin{frame}{目录}
    \vfill
    \tableofcontents[hideallsubsections]
    \vfill
\end{frame}

\AtBeginSection[ ]
{
\begin{frame}{目录}
    \vfill
    \tableofcontents[currentsection,hideallsubsections]
    \vfill
\end{frame}
}

\section{微分中值定理}

\subsection{费马定理}

\begin{frame}{费马定理}
    设函数$f(x)$在点$x_0$取得极值,且$f(x)$在$x_0$可导,则$f'(x_0)=0$.

    \begin{itemize}
        \item 导数值为$0$的点称为函数的\textbf{驻点}.
        \item 驻点一定是极值点吗?
    \end{itemize}
\end{frame}

\subsection{罗尔定理}

\begin{frame}{罗尔定理}
    设函数$f(x)$满足:
    \begin{itemize}
        \item 在闭区间$[a,b]$上连续
        \item 在开区间$(a,b)$上可导
        \item $f(a)=f(b)$
    \end{itemize}
    则$\exists \xi\in(a,b)$,使得
    \[
    f'(\xi)=0.
    \]
    \begin{itemize}
        \item 尝试用数学语言,一句话写出罗尔定理
    \end{itemize}
\end{frame}

\begin{frame}{推论}
    若$f(x)\in C[a,b]\cap D(a,b)$,且在开区间$(a,b)$内,$f'(x)\ne 0$,则$y=f(x)$在闭区间$[a,b]$上是单射函数,从而必存在反函数.
\end{frame}

\begin{frame}{例}
    设函数$f(x)\in C[a,b]$,在$(a,b)$内可导,且$f(a)=f(b)=0$,试证:
    \[
    \exists\ \xi\in (a,b),\ \text{s.t.}\ f'(\xi)=f(\xi)
    \]
\end{frame}

\begin{frame}{思路}
    设函数$f(x)\in C[a,b]$,在$(a,b)$内可导,且$f(a)=f(b)=0$,试证$\exists\ \xi\in (a,b),\ \text{s.t.}\ f'(\xi)=f(\xi)$.

    \begin{itemize}
        \item 导数与原函数之间的性质可以通过中值定理连接
        \item $f(a)=f(b)=0$,满足罗尔定理的使用条件,而$f'(\xi)-f(\xi)=0$
        \item 若能找到某个函数$F(x)$,使得$F'(x)=0 \Leftrightarrow f'(\xi)-f(\xi)=0$,得证
        \item $f'-f$并不像某个函数的导数,但可能是形如$f'g+fg'$进行约简后的结果\dots$\ \Rightarrow\ $
            找$g(x)$,使得$g'(x)=-g(x)$
        \item $g(x)=e^{-x}\Rightarrow F(x)=f(x)e^{-x}$.
    \end{itemize}
\end{frame}

\begin{frame}{练习}
    设函数$f(x)$在区间$[a,+\infty)$上连续,在开区间$(a,+\infty)$内可导,$a$是常数,$\lim\limits_{x\rightarrow +\infty}f(x)=f(a)$. 证明:
    \[
    \exists \ \xi\in (a,+\infty),\ \text{s.t.}\ f'(\xi)=0.
    \]
\end{frame}

\begin{frame}{思考题 1: 课本 例4.6}
    设$f(x)$在$[1,+\infty)$上可导,且$f(x)$有界,$f(1)=0$,证明:
    \[
    \exists\ \xi\in(1,+\infty),\ \text{s.t.}\ \xi f'(\xi)-f(\xi)=0.
    \]
    
    \begin{itemize}
        \item 提示:
        \begin{itemize}
            \item 题目只给了$f(1)=0$,而既没有给出某处$f(x_0)=0$,也没有给出$\lim\limits_{x\rightarrow \infty}f(x)=0$,
            反而是给出了$f(x)$有界. 这个条件要怎么利用?
            \item $\xi f'(\xi)-f(\xi)$看起来像某个函数的导数吗?不像的话,该怎么变换?
        \end{itemize}
    \end{itemize}
\end{frame}

\subsection{拉格朗日定理}

\begin{frame}{拉格朗日定理}
    设函数$f(x)$满足:
    \begin{itemize}
        \item 在闭区间$[a,b]$连续
        \item 在开区间$(a,b)$可导
    \end{itemize}
    则
    \[
    \exists\ \xi\in(a,b),\ \text{s.t.}\ f'(\xi)=\frac{f(b)-f(a)}{b-a}.
    \]
    不难看出,罗尔定理是拉格朗日定理的特殊形式.
\end{frame}

\begin{frame}{几何理解}
    \begin{figure}[H]
        \centering
        \includegraphics[width=0.5\textwidth]{lagrange.png}
        \caption{课本 P159 图4.4}
    \end{figure}
\end{frame}

\begin{frame}{证明}
    构造
    \[
    F(x)=f(x)-\frac{f(b)-f(a)}{b-a}(x-a)
    \]
    则
    \[
    F(a)=F(b)=f(a) \Rightarrow \exists\ \xi\in(a,b),\ F'(\xi)=0.
    \]
    怎么想到的?
\end{frame}

\begin{frame}{几何理解...不止于此!}
    一辆小车从$(a,f(a))$点出发,最终达到$(b,f(b))$点.
    \begin{itemize}
        \item 小车沿$AB$运动,则以上定理对所有$\xi\in(a,b)$成立.
        \item 小车不完全沿$AB$运动:
        \begin{itemize}
            \item 记小车在$y$方向上偏离直线$AB$的距离为$h(x)$.
            \item $h(x)$可能会随着小车行驶增加或减少,但一定有$h(a)=h(b)=0$.
            \item 同时,$h(x)$在小车行驶过程中一定会取到极(最)值:\textbf{最值定理}
        \end{itemize}
    \end{itemize}
    依照以上思路,可以完成拉格朗日定理的证明.
\end{frame}

\begin{frame}{其他形式}
    \[
    f(b)-f(a)=f'(\xi)(b-a)
    \]
    \[
    f(x_0+\Delta x)-f(x_0)=f'(x_0+\theta\Delta x)\Delta x,\quad\theta\in(0,1)
    \]
\end{frame}

\begin{frame}{推论}
    \begin{enumerate}
        \item 设函数$f(x)$在区间$I$上可导,且$f'(x)\equiv 0$,则$f(x)$在$I$上等于一常数.
        \item 设函数$f(x)$和$g(x)$在区间$I$上导数处处相等,则$f(x)$与$g(x)$在$I$上相差一常数.
    \end{enumerate}
\end{frame}

\begin{frame}{例}
    设函数$f(x)$在$[0,1]$上可导,且$0<f(x)<1$,又$\forall\ x\in(0,1)$,$f'(x)\ne1$,试证:在$(0,1)$内函数$f(x)$有\textbf{唯一}的不动点,即方程
    \[
    f(x)=x
    \]
    有\textbf{唯一}的实根.
\end{frame}

\begin{frame}{思考题 2: 课本 例4.11}
    设函数$f(x)$在区间$[a,+\infty)$上可导,且$\lim\limits_{x\rightarrow+\infty}f'(x)=A>0$,试证:
    \[
    \lim_{x\rightarrow +\infty}f(x)=+\infty.
    \]
    \begin{itemize}
        \item 提示:
        \begin{itemize}
            \item 命题在直觉上成立. 如何进行导数向函数值的变换?
            \item 当想不到能做些什么时,先尝试能如何将手边的条件利用起来:\\凑项,定义展开,\dots
        \end{itemize}
    \end{itemize}
\end{frame}

\subsection{柯西定理}

\begin{frame}{柯西定理}
    设函数$f(t)$和$g(t)$满足:
    \begin{itemize}
        \item 在闭区间$[a,b]$连续
        \item 在开区间$(a,b)$可导,且$\forall\ t\in(a,b),\ g'(t)\ne0$
    \end{itemize}
    则
    \[
    \exists\ \xi\in(a,b),\ \text{s.t.}\ \frac{f(b)-f(a)}{g(b)-g(a)}=\frac{f'(\xi)}{g'(\xi)}.
    \]
    不难看出,拉格朗日定理是柯西定理的特殊形式.
\end{frame}

\begin{frame}{几何理解}
    
    \begin{itemize}
        \item 记$x=g(t),y=f(t)$为以$t$为参数的参数方程定义的曲线.
        \item 等式左端等价于$(y_b-y_a)/(x_b-x_a)$,其几何意义为$A=(x_a,y_a),B=(x_b,y_b)$两点间连线的斜率
        \item 等式右端等价于$\frac{\mathrm dy}{\mathrm dt}/\frac{\mathrm dx}{\mathrm dt}=\frac{\mathrm dy}{\mathrm dx}$在$t=\xi$处的取值,
        其几何意义为$(x_\xi,y_\xi)$处的切线斜率
    \end{itemize}
    $\Rightarrow$割线与切线斜率相等
\end{frame}

\begin{frame}{例}
    设函数$f(x)$在$[a,b]$上连续,在$(a,b)$内可导$(a>0)$,证明:
    \[
    \exists\ \xi,\eta\in(a,b),\ \text{s.t.}\ abf'(\xi)=\eta^2f'(\eta)
    \]
    \begin{itemize}
        \item 提示:
        \begin{itemize}
            \item 找找$g(x)$?
        \end{itemize}
    \end{itemize}
\end{frame}

\subsection{导函数的性质}

\begin{frame}{达布定理}
    若函数$f(x)$在闭区间$[a,b]$上可导,且$f'_+(a)<f'_-(b)$,则
    \[
    \forall\ c\in(f'_-(a),f'_+(b)),\ \exists\ \xi\in(a,b),\ \text{s.t.}\ f'(\xi)=c.
    \]
    导函数即使不连续,也有介值性
\end{frame}

\begin{frame}{导函数的极限}
    设$\delta > 0$,函数$f(x)$在$[x_0,x_0+\delta)$上连续,在$(x_0,x_0+\delta)$内可导,若$\lim\limits_{x\rightarrow x_0^+}f'(x)=A$存在,则
    $f(x)$在点$x_0$有右导数,且$f_+'(x_0)=A$.

    \begin{itemize}
        \item 修改所有与右导数相关的内容为左导数,命题依然成立.
        \item 即,若函数在某点的单侧邻域内连续且可导,则该点的单侧导数存在且等于到函数的极限.
    \end{itemize}
\end{frame}

\section{洛必达法则}

\begin{frame}{洛必达法则}
    设函数$f(x)$和$g(x)$在点$x_0$的某个去心邻域$\dot U(x_0,\delta)$内有定义,且满足:
    \begin{itemize}
        \item $\lim\limits_{x\rightarrow x_0} f(x)=0,\lim\limits_{x\rightarrow x_0} g(x)=0$
        \item $f(x)$,$g(x)$在该去心邻域内可导,且$g'(x)\ne 0$
        \item $\displaystyle\lim_{x\rightarrow x_0}\frac{f'(x)}{g'(x)}=A$($A$为常数或$\infty$)
    \end{itemize}
    则
    \[
    \lim_{x\rightarrow x_0}\frac{f(x)}{g(x)}=\lim_{x\rightarrow x_0}\frac{f'(x)}{g'(x)}=A
    \]
\end{frame}

\begin{frame}{几何理解}
    \url{https://www.bilibili.com/video/BV1qW411N7FU/?p=7}
    \begin{figure}[H]
        \centering
        \includegraphics[width=0.9\textwidth]{3b1b.png}
    \end{figure}
\end{frame}

\begin{frame}{几何理解}
    \begin{itemize}
        \item 函数可导,意味着在该点附近,函数可以近似地被视为直线
        \item 取一小段距离$\mathrm dx$,得到函数值对应的变化$\mathrm df,\mathrm dg$
        \item $\mathrm df/\mathrm dg$可以作为一个$f(x)/g(x)$在该点附近的近似值
        \item 取$\mathrm dx\rightarrow 0$,极限即为该处$f(x)/g(x)$的极限值
    \end{itemize}
\end{frame}

\begin{frame}{思考}
    洛必达法则的第二项条件:
    \begin{itemize}
        \item $f(x)$,$g(x)$在该去心邻域内可导,且$g'(x)\ne 0$
    \end{itemize}
    那么,为什么在进行一次洛必达法则运用后,若依然得到$0/0$式,还可以继续运用?此时难道不是$g(x)=0$吗?
    \begin{itemize}
        \item 此时仅有$g(x_0)=0$,而$x_0$不在极限运算的定义域内.
    \end{itemize}
\end{frame}

\begin{frame}{洛必达法则}
    设函数$f(x)$和$g(x)$在点$x_0$的某个去心邻域$\dot U(x_0,\delta)$内有定义,且满足:
    \begin{itemize}
        \item $\lim\limits_{x\rightarrow x_0} g(x)=\infty$
        \item $f(x)$,$g(x)$在该去心邻域内可导,且$g'(x)\ne 0$
        \item $\displaystyle\lim_{x\rightarrow x_0}\frac{f'(x)}{g'(x)}=A$($A$为常数或$\infty$)
    \end{itemize}
    则
    \[
    \lim_{x\rightarrow x_0}\frac{f(x)}{g(x)}=A
    \]
\end{frame}

\begin{frame}{例}
    求
    \begin{align*}
        \lim_{x\rightarrow -\infty}\frac{x}{\sqrt{1+x^2}}
    \end{align*}
\end{frame}

\begin{frame}{“典”例}
    \begin{align*}
        \lim_{x\rightarrow 0}\frac{x^2\sin\frac1x}{\sin x}&=\lim_{x\rightarrow 0}\frac{2x\sin\frac1x-\cos\frac1x}{\cos x}\\
        &=\lim_{x\rightarrow 0}\frac{2x\sin\frac1x}{\cos x}-\lim_{x\rightarrow 0}\frac{\cos\frac1x}{\cos x}
    \end{align*}
    第一项极限为$0$,第二项极限不存在,故该极限不存在.
\end{frame}

\begin{frame}{投票}
    
\end{frame}

\begin{frame}{使用原则}
    \begin{itemize}
        \item 洛必达法则的本质:
        \begin{itemize}
            \item 函数在$x=x_0$处的比值\ 等价于\\ 函数在$\dot U(x_0)$内近似直线斜率的比值在$x\rightarrow x_0$时的极限
            \item 与泰勒公式等价
        \end{itemize}
        \item 使用原则:
        \begin{itemize}
            \item 使用前先尽可能化简
            \item 确保条件均成立:\\0/0或$\infty/\infty$,去心邻域内可导,上下求导后极限存在
        \end{itemize}
    \end{itemize}
\end{frame}

\section{泰勒公式}

\begin{frame}{泰勒定理 1}
    设函数$f(x)$在点$x_0$的邻域内有定义,且在$x_0$有$n$阶导数,那么
    \[
    f(x)=f(x_0)+f'(x_0)(x-x_0)+\cdots+\frac{f^{(n)}(x_0)}{n!}(x-x_0)^n+o((x-x_0)^n)
    \]
    其中$o((x-x_0)^n)$称为佩亚诺余项,定理结论称为$f(x)$的带佩亚诺余项的$n$阶泰勒公式.
    
    一个确定函数的泰勒公式中,多项式的系数是确定的.
\end{frame}

\begin{frame}{几何意义}
    对$e^x$进行$0$处的多项式估计
    \[
    e^x=1+x+\frac{1}{2}x^2+\frac{1}{3!}x^3+\frac{1}{4!}x^4+\cdots+\frac{1}{n!}x^{n}+o(x^n)
    \]
\end{frame}

\begin{frame}{泰勒定理 2}
    设函数$f(x)$在包含点$x_0$的开区间$(a,b)$内具有$n+1$阶导数,则$\forall x\in(a,b)$,有 
    \[
    f(x)=f(x_0)+f'(x_0)(x-x_0)+\cdots+\frac{f^{(n)}(x_0)}{n!}(x-x_0)^n+\frac{f^{(n+1)}(\xi)}{(n+1)!}(x-x_0)^{n+1}
    \]
    其中$\xi$介于$x_0$与$x$之间.

    $\displaystyle\frac{f^{(n+1)}(\xi)}{(n+1)!}(x-x_0)^{n+1}$称为拉格朗日余项.
    
\end{frame}

\begin{frame}{麦克劳林公式}
    在$0$处的泰勒公式. 课本P176$\sim$177
\end{frame}

\begin{frame}{例 (23Fall, Midterm)}
    求$\sqrt{1+\sin x}$带佩亚诺余项的3阶麦克劳林公式.
\end{frame}

\begin{frame}{思考题 3:课本 例4.36}
    设$f(x)$在$[0,1]$上二阶可导,且$\max\limits_{0<x<1}f(x)=1/4$,$|f''(x)|\le1$,试证:
    \[
    |f(0)|+|f(1)|<1.
    \]
    \begin{itemize}
        \item 提示:
        \begin{itemize}
            \item 当你没有思路的时候,尝试利用手边的条件做点什么(随便什么)
            \item 要证$|f(0)|+|f(1)|<1$,需要对函数值的绝对值进行限制. 题目给定的条件要如何与函数值联系起来?
        \end{itemize}
    \end{itemize}
\end{frame}

% -------------------------------------------------------
% \section*{}
% \begin{frame}{反馈问卷}
%     \begin{figure}[H]
%         \centering
%         \includegraphics[width=0.5\linewidth]{qrcode.png}
%     \end{figure}
% \end{frame}
% \begin{frame}
% \vspace{25pt}
% \[
% \text{\Huge }
% \]
% \end{frame}

\end{document}
\documentclass[12pt,addpoints]{exam}

\usepackage{xeCJK}
\usepackage{fontspec}
\usepackage{amsmath,amssymb}
\usepackage{graphicx}
\usepackage{tabularx}

\title{
    \vfill
    \LARGE ShanghaiTech University \\
    \bigskip
    \textbf{GEMA1001 Calculus I} \\
    \textbf{Fall 2023}   \\
    \bigskip
    Midterm Exam
    \author{Answer composed by Yixuan Liu}
    \date{November 16, 2025}
    \vfill
}

\qformat{\bf\thequestion. (\totalpoints\ pts) \thequestiontitle\hfill}
\pointname{'}
\CorrectChoiceEmphasis{\bf\color{blue}}

\printanswers

\renewcommand{\baselinestretch}{1.3}
\setlength{\parskip}{1.25\baselineskip}

\begin{document}

\maketitle

\begin{center}
    If you have any questions about the answers, feel free to contact through email(liuyx2023@shanghaitech.edu.cn) or QQ(2987221272).
\end{center}
\thispagestyle{empty}

\begin{questions}

\newpage

\titledquestion{单项选择题}

\begin{parts}
    \part[4] 已知函数$f(x)=|x|$,$g(x)=\text{sgn}(x)$,则$g(f(x))=$($\qquad$)
    
    \begin{oneparchoices}
        \choice 
        $\left\{
        \begin{array}{ll}
           -1&x\ne0\\
           0&x=0.
        \end{array}
        \right.$

        \choice 
        $\left\{
        \begin{array}{ll}
           1&x\ne0\\
           0&x=0.
        \end{array}
        \right.$

        \choice 
        $\left\{
        \begin{array}{ll}
           -1&x\ne0\\
           1&x=0.
        \end{array}
        \right.$

        \choice 
        $\left\{
        \begin{array}{ll}
           1&x\ne0\\
           -1&x=0.
        \end{array}
        \right.$
    \end{oneparchoices}

    \begin{solution}
        B.
        
        $f(x)=|x|\ge 0$,因此在所有非零点,$g(f(x))=g(|x|)=1$. 注意$g(x)$在$0$处的定义为$0$,
    \end{solution}

    \part[4] 设函数$f(x)$在$\mathbb R$上有定义,且对$\forall x\in\mathbb R$,均有$x+\cos x\le f(x) \le x+1$,则$f(x)$在点$x=0$处($\qquad$)

    \begin{tabularx}{\textwidth}{X X}
    A. 不存在极限. & B. 有极限但不连续. \\
    C. 连续但不可导. & D. 可导. \\
    \end{tabularx}

    \begin{solution}
        D.

        首先,根据夹逼定理,函数$f(x)$在$x=0$处的极限存在:
        \[
        1=\lim_{x\rightarrow0}(x+\cos x) \le \lim_{x\rightarrow 0}f(x)\le \lim_{x\rightarrow 0}(x+1)=1.
        \]
        对不等式直接取$x=0$,得$1\le f(0)\le 1$,因此$f(0)=1=\lim\limits_{x\rightarrow 0}f(x)$. 故$f(x)$在点$x=0$处连续.
        
        假设$f(x)$在$0$处可导,根据导数的定义:
        \[
        f'(0)=\lim_{x\rightarrow 0}\frac{f(x)-f(0)}{x-0}=\lim_{x\rightarrow0}\frac{f(x)-1}x
        \]
        据不等式
        \[
        x+\cos x - 1\le f(x)-1\le x
        \]
        当$x>0$,有$\displaystyle1+\frac{\cos x - 1}{x}\le \frac{f(x)-1}{x} \le 1$. 据夹逼定理,$\displaystyle\lim_{x\rightarrow 0^+}\frac{f(x)-1}{x}=1$. 
        
        \vspace{5pt}
        故$f(x)$的右导数存在. 同理可得左导数也存在,且左右导数相等,故$f(x)$可导.
    \end{solution}

    
    \part[4] 若曲线$C$的参数方程为$\left\{\begin{array}{c}
        x=2\sin t+\cos t,\\ y=\sin t -2\cos t,
    \end{array}\right.$
    则$C$在$t=0$处的切线方程为($\qquad$)

    \vspace{5pt}

    \begin{tabularx}{\textwidth}{X X}
        A.$\displaystyle y=\frac12x-\frac52$.& B.$\displaystyle y=2x-4.$\\
        \vspace{5pt}C.$\displaystyle y=-\frac12x-\frac32$. & \vspace{5pt} D.$\displaystyle y=-2x.$
    \end{tabularx}

    \begin{solution}
        A.
        
        首先求$t=0$时的点:$x=1, y=-2$. 根据参数方程求导法则,
        \[
        \frac{\mathrm dx}{\mathrm dt}=2\cos t-\sin t,\quad \frac{\mathrm dy}{\mathrm dt}=\cos t +2\sin t.
        \]
        代入$t=0$得$\frac{\mathrm dx}{\mathrm dt}=2,\ \frac{\mathrm dy}{\mathrm dt}=1$. 故$\frac{\mathrm dy}{\mathrm dx}=\frac12.$
        
        将点$(1,-2)$和斜率$\frac12$代入直线方程,得答案为A.
    \end{solution}

    \part[4] 设函数$f(x)$在$\mathbb R$上可导,则“$f(x)$在$\mathbb R$上有界”是“$\displaystyle\lim_{x\rightarrow \infty}f'(x)$存在”的($\qquad$)
    
    \begin{tabularx}{\textwidth}{X X}
        A.充分但非必要条件. & B.必要但非充分条件.\\
        C.充分且必要条件. & D.既非充分又非必要条件.
    \end{tabularx}

    \begin{solution}D.

        反例:$\sin x$在$\mathbb R$上可导且有界,但$x\rightarrow\infty$时导函数极限不存在.

        $f(x)=x$在$\mathbb R$上可导且$x\rightarrow \infty$时$f'(x)\equiv 1$极限存在,但其在$\mathbb R$上无界.
        
    \end{solution}

    \part[4] 设函数$f(x)$在$\mathbb R$上连续,$a_1\in\mathbb R$,$a_{n+1}=f(a_n)$,$n\in\mathbb Z^+$. 关于下列两个结论:
    \begin{enumerate}
        \item [(1)] 若$f(x)$严格单调增加且有上界,则数列$\{a_n\}$收敛;
        \item [(2)] 若$f(x)$严格单调减少且有界,则数列$\{a_n\}$收敛;
    \end{enumerate}
    正确的选项是($\qquad$)
    
    \begin{tabularx}{\textwidth}{X X}
        A. (1) (2) 都正确 & B. (1) (2)都错误.\\
        C. (1)正确 (2)错误. & D. (1)错误 (2)正确.
    \end{tabularx}

    

    \begin{solution} B.
        
        对(1),考虑$a_{k+1}=f(a_k)$和$a_k$的关系.
        \begin{itemize}
            \item 若$a_{k+1}>a_k$,则$a_{k+2}=f(a_{k+1})>f(a_k)=a_{k+1}$. 由数学归纳法易得数列$\{a_n\}$单调递增,又$\{a_n\}$有上界,故$a_n$收敛;
            \item 若$a_{k+1}<a_k$,则$a_{k+1}=f(a_{k+1})<f(a_k)=a_{k+1}$. 由数学归纳法易得数列$\{a_n\}$单调递减,而无下界,则$\{a_n\}$不一定收敛;
            \item 若$a_{k+1}=a_k$,则显然数列$\{a_n\}$收敛.
        \end{itemize}
        因此,(1)错误.

        对(2),考虑函数$f(x)=-x+1$,取$a_1=1$,则$a_k$的值持续在$0$与$1$之间震荡,极限不存在,故(2)错误
    \end{solution}
\end{parts}

\newpage

\titledquestion{填空题}

\begin{parts}
    \vspace{5pt}

    \part[4] 极限$\displaystyle\lim_{n\rightarrow \infty}\left(\frac{n+1}{n^2+1}+\frac{n+2}{n^2+2}+\cdots+\frac{n+n}{n^2+n}\right)=\underline{\qquad\qquad\qquad\qquad\qquad\qquad}.$
    
    \begin{solution}$3/2.$
        
        考虑放缩
        \[
        \sum_{i=1}^{n}\frac{n+i}{n^2+n}\le \sum_{i=1}^n \frac{n+i}{n^2+i}\le \sum_{i=1}^{n}\frac{n+i}{n^2}
        \]
        则
        \[
        \sum_{i=1}^{n}\frac{n+i}{n^2+n}=\frac{n^2}{n^2+n}+\frac{n(n+1)}{2(n^2+n)}
        \]
        当$n\rightarrow \infty$,该式极限为$\frac32$. 同理,右式极限也为$\frac32$. 由夹逼定理,原极限为$\frac32$.

    \end{solution}

    \part[4] 函数$\displaystyle f(x)=\frac{\ln |x|}{x-2}\sin\frac{1}{x-1}$的第二类间断点的个数是:$\underline{\qquad\qquad\qquad\qquad}.$

    \begin{solution}2.
        
        $f(x)$的定义域为:
        \[
        |x|>0,\ x-2\ne 0,\ x-1\ne 0 \Leftrightarrow x\in \mathbb R \backslash \{0, 1, 2\}.
        \]
        \begin{itemize}
            \item $0$是无穷间断点,$\lim\limits_{x\rightarrow 0}f(x)=\lim\limits_{x\rightarrow 0}\ln|x|\cdot\frac{\sin (-1)}{-2}$,有界量乘以无穷大量仍是无穷大量.
            \item $1$是可去间断点,$\lim\limits_{x\rightarrow 1}|f(x)|\le\lim\limits_{x\rightarrow 1}|-\ln|x||=0$,左右极限存在且相等.
            \item $2$是无穷间断点,$\lim\limits_{x\rightarrow 2}f(x)=\lim\limits_{x\rightarrow 2}\ln2\cdot\sin1\cdot(x-2)^{-1}$,有界量乘以无穷大量仍是无穷大量.
        \end{itemize}
    \end{solution}

    \part[4] 已知函数$f(x)=x^{\sin x}$,则$\mathrm df\Big|_{x=\frac\pi2}=\underline{\qquad\qquad\qquad\qquad\qquad\qquad\qquad\qquad}.$

    \begin{solution}$\mathrm dx$.
        
        法1:\[
        f'(x)=(e^{\ln x\cot\sin x})'=e^{\ln x\cdot \sin x}\cdot (\sin x/x +\ln x\cos x)
        \]
        故
        \[
        f'(\frac\pi 2)=(\frac\pi 2)^{1}\cdot(1/(\frac\pi2)+0)=1.
        \]

        法2:
        \[
        \ln f(x)=\sin x\cdot\ln x \Leftrightarrow \frac{f'(x)}{f(x)}=\cos x\cdot \ln(x)+\frac{\sin x}{x}
        \]
        故
        \[
        f'(\frac{\pi}{2})=f(\frac\pi2)(\cos\frac{\pi}{2}\cdot \ln\frac\pi 2+\sin\frac\pi2\cdot \frac2\pi)=\frac\pi2\cdot(0+\frac2\pi)=1.
        \]

        因此结果为$1\mathrm dx=\mathrm dx$.

    \end{solution}

    \part[4] 若可微函数$y=y(x)$由方程$y=-ye^x+2e^y\sin x-7x$所确定,则$y'(0)=\underline{\qquad\qquad\qquad}.$

    \begin{solution}$-5/2$.
        
        两边求导
        \[
        y'=-(y'e^x+ye^x)+2(y'e^y\sin x+e^y\cos x)-7
        \]

        将$x=0$代回原方程
        \[
        y|_{x=0}=-y|_{x=0}e^0+2e^y|_{x=0}\sin 0-7\cdot 0=-y|_{x=0} \Leftrightarrow y|_{x=0}=0.
        \]

        因此(以下省略$x=0$)
        \[
        y'=-(y'e^0+ye^0)+2(y'e^y\sin 0+e^y\cos 0)-7=-(y'+0)+2(0+1)-7
        \]
        即$\displaystyle y'(0)=-\frac{5}{2}$.

    \end{solution}

    \part[4] 曲线$x=y^4+2y^3-1$在点$(2,1)$处的法线方程为$\underline{\qquad\qquad\qquad\qquad\qquad}$.

    \begin{solution} $y=-10x+21$.
        
        \[
        \frac{\mathrm dx}{\mathrm dy}=4y^3+6y^2 \Leftrightarrow \frac{\mathrm dy}{\mathrm dx}=\frac{1}{4y^3+6y^2}.
        \]
        代入$x=2,y=1$得
        \[
        \frac{\mathrm dy}{\mathrm dx}\Big|_{x=2}=\frac{1}{4+6}=\frac{1}{10}.
        \]
        故切线斜率为$1/10$,则法线斜率为$-10$. 因此法线方程为
        \[
        y-1=-10(x-2)\Leftrightarrow y=-10x+21.
        \]
    \end{solution}
\end{parts}

\titledquestion{极限定义证明题}[8]

    用极限定义证明:$\displaystyle\lim_{x\rightarrow\infty}\frac{2x^2}{x^2-x+1}=2.$

    \begin{solution}
        已知,当$|x|>1$时,
        \[
        \Big|\frac{2x^2}{x^2-x+1}-2\Big|=\Big|\frac{2x-2}{x^2-x+1}\Big|<2|\frac{x-1}{x^2-x}|=\frac{2}{|x|}
        \]

        取$\displaystyle X=\max\Big\{1,\frac{2}{\epsilon}\Big\}$,则

        \[
        \forall \varepsilon > 0, |x| > X, \Big|\frac{2x^2}{x^2-x-1}-2\Big|<\frac2{|x|}<2\cdot \frac{\varepsilon}{2}=\varepsilon
        \]
        即得结果,证毕.$\hfill\square$
    \end{solution}
    
\titledquestion{极限计算}

\begin{parts}
    \part[8] 求极限$\displaystyle \lim_{x\rightarrow 0}\left(\frac{\ln(1+x)}{x}\right)^{\frac{1}{e^x-1}}$.

    \begin{solution}
        \begin{align*}
            \lim_{x\rightarrow 0}\left(\frac{\ln(1+x)}{x}\right)^{\frac{1}{e^x-1}}&=\lim_{x\rightarrow 0}e^{\ln(\frac{\ln(1+x)}{x})\frac{1}{e^x-1}}
        \end{align*}
        问题转化为求$\displaystyle\lim_{x\rightarrow 0}\ln(\frac{\ln(1+x)}{x})\frac{1}{e^x-1}$.
        \begin{align*}
            \lim_{x\rightarrow 0}\ln(\frac{\ln(1+x)}{x})\frac{1}{e^x-1}&=\lim_{x\rightarrow 0}\ln(\frac{x-\frac12x^2+o(x^2)}{x})\frac{1}{e^x-1}\\
            &=\lim_{x\rightarrow 0}\ln(1-\frac12x+o(x))\frac{1}{e^x-1}\\
            &=\lim_{x\rightarrow 0}\frac{-\frac12x+o(x)}{x+o(x)}=-\frac12.
        \end{align*}
        故原极限
        \begin{align*}
            \lim_{x\rightarrow 0}\left(\frac{\ln(1+x)}{x}\right)^{\frac{1}{e^x-1}}&=\lim_{x\rightarrow 0}e^{\ln(\frac{\ln(1+x)}{x})\frac{1}{e^x-1}}\\
            &=e^{\lim\limits_{x\rightarrow 0}\ln(\frac{\ln(1+x)}{x})\frac{1}{e^x-1}}\\
            &=e^{-\frac12}=\frac{1}{\sqrt{e}}.
        \end{align*}
    \end{solution}

    

    \part[8] 已知数列$\{a_n\}$满足$a_1\in(0,1]$,$\displaystyle a_{n+1}=\frac{a_n+a_n^3}{2}$,$n=1,2,3,\cdots$. 
    证明$\lim\limits_{n\rightarrow\infty}a_n$存在并求之.

    \begin{solution}
        先使用数学归纳法证明$a_n$有界. 已知$a_1\in(0,1]$,假设$a_k\in(0,1]$. 那么:
        \[
        a_{k+1}=\frac12a_k(a_k^2+1)\le\frac{1}{2}\cdot 2=1;\quad \text{显然}a_{k+1}>0
        \]
        因此,$\forall n\in N^+$,$a_n\in(0,1]$. 另,
        \[
        a_{k+1}-a_k=\frac{a_k+a_k^3}{2}-a_k=\frac{a_k^3-a_k}{2}=\frac12a_k(a_k^2-1)\le0.
        \]
        因此,数列$\{a_n\}$单调递减有下界,极限存在. 设极限为$L$,则
        \[
        L = \frac{L+L^3}{2}
        \]
        解得$L=0$或$L=1$. 可以看出,当$a_1\ne1$时,极限为$0$;当$a_1=1$时,极限为$1$.
    \end{solution}

\end{parts}

\titledquestion{导数计算}

\begin{parts}
    \part[8] 已知函数$\displaystyle f(x)=\left\{\begin{array}{ll}
        \displaystyle\ln(1+x^2),&x\le 0,\\
        \displaystyle x^2\sin\frac1x,&x>0,
    \end{array}\right.$ 求$f'(x)$.

    \begin{solution}
        $x=0$为该函数的间断点. 由于该点左右极限都存在且相等$0$,函数是$\mathbb R$上的连续函数.
        \begin{itemize}
            \item $x<0$,$\displaystyle f'(x)=\frac{2x}{1+x^2}$.
            \item $x>0$,$\displaystyle f'(x)=2x\sin\frac1x+x^2\cos\frac1x\cdot(-\frac{1}{x^2})$
        \end{itemize}
        导函数在$x=0$以外的所有点连续. 考虑$x=0$,该处的左导数为$\lim\limits_{x\rightarrow0^-}\frac{2x}{1+x^2}=0$,右导数$\lim\limits_{x\rightarrow 0^+}f'(x)=\lim\limits_{x\rightarrow 0^+}-\cos \frac1x$,极限不存在.

        故函数$f(x)$在$x=0$处的导数不存在. 综上所述,
        \[
        f'(x)=\left\{\begin{array}{ll}
            \displaystyle \frac{2x}{1+x^2}, &x<0\\
            \displaystyle2x\sin\frac1x-\cos\frac1x, &x>0.
        \end{array}\right.
        \]
    \end{solution}

    

    \part[8] 设函数$f(x)=(x+1)^2\ln x$,求$f^{(n)}(2)$($n=1,2,3,\cdots$).

    \begin{solution}
        根据Leibniz公式,当$n\ge 3$,
        \begin{align*}
            f^{(n)}(x)&=\sum_{k=0}^{n}{n\choose k}(x^2+2x+1)^{(k)} (\ln x)^{(n-k)}\\
            &=\sum_{k=0}^{2}{n\choose k}(x^2+2x+1)^{(k)} (\ln x)^{(n-k)}+\sum_{k=3}^{n}{n\choose k}(x^2+2x+1)^{(k)} (\ln x)^{(n-k)}\\
            &=\sum_{k=0}^{2}{n\choose k}(x^2+2x+1)^{(k)} (\ln x)^{(n-k)}+0\\
            &=(x^2+2x+1)\cdot(-1)^{n+1}(n-1)!x^{-n}+n\cdot2(x+1)(-1)^{n}(n-2)!x^{-(n-1)}\\
            &\quad +n(n-1)(-1)^{n-1}(n-3)!x^{-(n-2)}
        \end{align*}

        当$n=1$,$f'(x)=2(x+1)\ln x+(x+1)^2\frac1x$. 综上所述,

        \[
        f^{(n)}(x)=\left\{
            \begin{array}{ll}
                \displaystyle (2x+2)\ln x + \frac{x^2+2x+1}{x}\vspace{5pt},&n=1\vspace{5pt}\\
                \displaystyle -\frac{x^2+2x+1}{x^2}+\frac{4x+4}{x}+2\ln x, &n=2\vspace{5pt}\\
                \displaystyle (x^2+2x+1)\cdot(-1)^{n+1}(n-1)!x^{-n}\\ \quad+n\cdot2(x+1)(-1)^{n}(n-2)!x^{-(n-1)}\\
                    \quad +n(n-1)(-1)^{n-1}(n-3)!x^{-(n-2)}, &n\ge 3.
            \end{array}
        \right.
        \]
        \[
        f^{(n)}(2)=\left\{
            \begin{array}{ll}
                \displaystyle6\ln2+\frac92,\vspace{5pt}&n=1\vspace{5pt}\\
                \displaystyle2\ln2+\frac{15}{4},\vspace{5pt}&n=2\vspace{5pt}\\
                \displaystyle(-1)^n\frac{n!}{2^n}\Big(-\frac9n+\frac{12}{n-1}-\frac{4}{n-2}\Big),&n\ge 3
            \end{array}
        \right.
        \]
    \end{solution}
\end{parts}

\ifprintanswers
\newpage
\fi

\titledquestion{解答题}

\begin{parts}
    \part[2] 写出$(1+x)^\alpha$($\alpha\in\mathbb R$)带佩亚诺余项的$n$阶麦克劳林公式.

    \begin{solution}
        \[
        (1+x)^\alpha = 1+\alpha x+\frac{\alpha (\alpha-1)}{2!}x^2+\cdots+\frac{\alpha(\alpha-1)\cdots(\alpha-n+1)}{n!}x^n+o(x^n)
        \]
        \textbf{注意:由于$\alpha$不一定是正整数,不可以写成阶乘.}
    \end{solution}

    
    \part[8] 求$\sqrt{1+\sin x}$带佩亚诺余项的3阶麦克劳林公式.

    \begin{solution}

        记$f(x)=\sqrt{1+\sin x}$.$\qquad f(0)=\sqrt{1+0}=1$. 

        $f'(0)=[\frac12(1+\sin x)^{-1/2}\cos(x)]|_{x=0}=\frac12$.

        $f''(0)=-\frac14(1+\sin x)^{-3/2}\cos x - \frac12(1+\sin x)^{-1/2}\sin x|_{x=0}=-\frac14.$
        
        $f^{(3)}(0)=\frac{3}{8}(1+\sin x)^{-5/2}\cos x+0+0-\frac12(1+\sin x)^{-1/2}\cos x|_{x=0}=-\frac18$.
        
        故
        \[
        \sqrt{1+\sin x}=1+\frac12x-\frac18x^2-\frac{1}{48}x^3+o(x^3).
        \]
        
    \end{solution}
\end{parts}

\ifprintanswers
\newpage
\fi

\titledquestion{证明题}
    已知函数$\displaystyle g(x)=\left\{
        \begin{array}{ll}
            \displaystyle \frac{\sin \pi x}{x(x-1)},&x\ne0, 1,\\
            2,&x=0,1.
        \end{array}
    \right.$ 证明:
\begin{parts}
    \part[4] $x=0$和$x=1$是$g(x)$的可去间断点.

    \begin{solution}
        \begin{align*}
            \lim_{x\rightarrow 0}\frac{\sin \pi x}{x(x-1)}&=-\lim_{x\rightarrow0}\frac{\pi\sin \pi x}{\pi x}=-\pi;\\
            \lim_{x\rightarrow 1}\frac{\sin \pi x}{x(x-1)}&=\lim_{x\rightarrow1}\frac{-\pi\sin \pi (x-1)}{\pi(x-1)}=-\pi.\\
        \end{align*}
        因此$x=0$和$x=1$是$g(x)$的可去间断点.
    \end{solution}
    
    \part[6] 若函数$f(x)$在$\mathbb R$上可导,且$f(x)g(x)$连续,则存在$\xi\in\mathbb R$,使得$1+\xi f'(\xi)=e^{-f(\xi)}$.

    \begin{solution}
        首先分析$f(x)$的性质. 不难发现,$g(x)$在$x=0,x=1$两点是不连续的, 在其它点是连续的,因此$f(x)g(x)$在$x=0,x=1$两点外也是必然连续的.
        要使$f(x)g(x)$在$x=0,x=1$处连续,需有
        \[
        \lim_{x\rightarrow x_0}f(x)g(x)=f(0)g(0),\quad x_0=0\text{或}x_0=1.
        \]
        则,因为$f(x)$连续,$g(x)$在$x=0,x=1$两点的极限存在,
        \[
        \lim_{x\rightarrow x_0}f(x)g(x)=\lim_{x\rightarrow x_0}f(x)\lim_{x\rightarrow x_0}g(x)=f(x_0)\cdot(-\pi)=2f(x_0)
        \]
        因此$f(x_0)=0$,即$f(0)=f(1)=0$.

        构造辅助函数$F(x)=xe^{f(x)}$. $F(0)=0,F(1)=1$. 则
        \[
        \exists \xi\in(0,1),\ \text{s.t.} F'(\xi)=\frac{F(1)-F(0)}{1-0}=1.
        \]
        而
        \[
        F'(x)=e^{f(x)}+xe^{f(x)}f'(x)
        \]
        因此
        \[
        \exists \xi\in(0,1),\ F'(\xi)=e^{f(\xi)}(1+\xi f'(\xi))\ \text{即}\ 1+\xi f'(\xi)=e^{-f(\xi)}.
        \]$\hfill\square$
    \end{solution}

\end{parts}


\end{questions}

\end{document}

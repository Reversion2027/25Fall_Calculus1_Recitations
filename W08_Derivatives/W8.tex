\documentclass[]{beamer}
\usepackage[utf8]{inputenc}
\usepackage{xeCJK}
\usepackage{graphicx}
\usepackage{subfigure}
\usepackage{mathtools}
\usepackage{utopia} %font utopia imported
\usetheme{CambridgeUS}
\usecolortheme{dolphin}
\usefonttheme{professionalfonts}
\usepackage{natbib}
\usepackage{hyperref}
\usepackage{fontspec}
\usepackage{setspace}
\usepackage{float}
\usepackage{extarrows}
% \usepackage{enumitem}

\setCJKmainfont{SourceHanSansSC-Regular.otf}[Path=../fonts/, BoldFont=bold.otf]

\setbeamerfont{title}{size=\Large}
\setbeamerfont{subtitle}{size=\small}
\setbeamerfont{date}{size=\small}
\setbeamerfont{institute}{size=\small}

\setstretch{1.3}
% \setlength{\parindent}{2em}
% \setlength{\parskip}{0pt}

% \setlist[itemize]{leftmargin=2em}

% ↓↓↓ Modify this ↓↓↓
\title{高等数学I\quad 习题课08}
\subtitle{闭区间上的连续函数,导数}
\date[2025.11.6]{2025.11.6}
% ↑↑↑ Modify this ↑↑↑

\author[上海科技大学]{}
\institute[]{上海科技大学}

\begin{document}

\begin{frame}
    \vspace{15pt}
    \titlepage
\end{frame}

\begin{frame}{Quiz}
    \[
    \text{\Huge 18:00 - 18:40}
    \]
\end{frame}

\begin{frame}{习题课07 反馈}
    \begin{columns}
        % 左栏:文字
        \begin{column}{0.5\textwidth}
            \begin{figure}[H]
                \centering
                \includegraphics[width=1.0\linewidth]{fb1.png}
                \caption{课程质量}
            \end{figure}
        \end{column}

        \begin{column}{0.5\textwidth}
            \begin{figure}[H]
                \centering
                \includegraphics[width=1.0\linewidth]{fb2.png}
                \caption{课堂氛围}
            \end{figure}
        \end{column}
    \end{columns}
\end{frame}

\begin{frame}{习题课07 反馈}
    \begin{itemize}
        \item 找一些往年题目进行讲解,熟悉考试风格
        \begin{itemize}
            \item 目前学到的内容太少,暂时找不到合适的题目
        \end{itemize}
        \item 证明题多留点时间思考
        \begin{itemize}
            \item OK
        \end{itemize}
    \end{itemize}
\end{frame}

\begin{frame}{目录}
    \tableofcontents
\end{frame}

\AtBeginSection[ ]
{
\begin{frame}{目录}
    \tableofcontents[currentsection]
\end{frame}
}

\section{闭区间上的连续函数}

\begin{frame}{性质}
    若$f\in C[a,b]$,则:
    \begin{enumerate}
        \item 【有界性定理】$f$在$[a, b]$上有界
        \item 【最大值最小值定理】$\exists\xi,\eta\in[a,b],\ \text{s.t.} \displaystyle f(\xi)=M=\max_{x\in[a,b]}f(x),f(\eta)=m=\min_{x\in[a,b]}f(x)$
        \item 【介值定理】\\
        若$f(a)\ne f(b)$,则$\forall c\in[f(a),f(b)],\exists\xi\in[a, b]\ \text{s.t. }f(\xi)=c.$
    \end{enumerate}
    \textbf{连续函数将闭区间映射为闭区间}
\end{frame}

\begin{frame}{例 (习题2 补充题 7)}
    设函数$f(x)\in C(\mathbb{R})$,证明:$f(x)$在$\mathbb{R}$上取到它的最小值.
\end{frame}

\section{导数}

\begin{frame}{引}
    \vfill

    \quad So far as the theories of mathematics are about reality,
    they are not certain; so far as they are certain, they are not about reality.
    \vspace{5pt}
    \qquad\qquad\qquad\qquad\qquad\qquad\qquad\qquad\qquad\qquad\qquad\qquad-- Albert Einstein


    \vfill
    \begin{flushright}
        $\scriptscriptstyle{\text{Borrowed from 3Blue1Brown, Essence of Calculus}}$
    \end{flushright}
\end{frame}

\begin{frame}{思考}
    \begin{itemize}
        \item 导数在实际应用中的含义是?
        \item “瞬时变化率”?
    \end{itemize}
\end{frame}

\begin{frame}{例}
    \begin{itemize}
        \item 对于一个物理问题,考察:
        \item 距离函数$s(t)$与速度函数$v(t)$的关系
        \item \dots 速度函数$v(t)$?
        \item “速度”作为一个变化量,为何可以在只有一个时刻的信息的前提下,得到具体的值?
    \end{itemize}
\end{frame} 

\begin{frame}{例}
    \begin{itemize}
        \item 汽车仪表盘能测量“瞬时速度”吗?显然不能。
        \item 真实世界中,能做的只有:
        \begin{itemize}
            \item 记录两组数据$(t_1, s_1),(t_1+\mathrm dt, s_1+\mathrm ds)$
            \item 其中,$\mathrm dt$是一个很小的时间间隔
            \item 计算出长度为$\mathrm dt$的时间间隔中的平均速率
        \end{itemize}
        \item 不妨试试用到函数分析当中?
    \end{itemize}
\end{frame}

\begin{frame}{例}
    考察函数$f(x)=x^2$\dots    
\end{frame}

\begin{frame}{Reality}
    \begin{itemize}
        \item 在现实中,我们无法令$\mathrm dt$的值小到完全可忽略;我们能做的仅有进行有限范围内的近似
        \[
        \frac{\mathrm ds}{\mathrm dt}(t)=\frac{s(t+\mathrm dt)-s(t)}{\mathrm dt}
        \]
        $\mathrm dt$不是\textbf{无穷小};$\mathrm dt$也不是$0$.
    \end{itemize}
\end{frame}

\begin{frame}{Theory}
    \begin{itemize}
        \item 但我们可以令$\mathrm dt$非常接近$0$.
        \[
        \frac{\mathrm ds}{\mathrm dt}(t)=\lim_{\mathrm dt\rightarrow 0}\frac{s(t+\mathrm dt)-s(t)}{\mathrm dt}
        \]
        \item 如此,在某个时刻的“变化率”就有了含义.
        \item 切线斜率?某一点附近的最佳直线近似!\\(回顾:习题课07 泰勒公式的几何理解)
    \end{itemize}
\end{frame}

\begin{frame}{Theory}
    \begin{itemize}
        \item 我们可以用数学的方法计算出这个极限的值:
        \[
        \frac{\mathrm ds}{\mathrm dt}(t)=\lim_{\mathrm dt\rightarrow 0}\frac{s(t+\mathrm dt)-s(t)}{\mathrm dt}
        \]
        \item 因此,导数不是一个简单的分数,而是当我们选择的$\mathrm{d}t$值无限逼近于$0$时,这个比值的极限.
        \item 从图像上如何理解?
    \end{itemize}
\end{frame}

\begin{frame}{Note}
    \begin{itemize}
        \item $\mathrm dt, \mathrm{d}s$都是\textbf{有大小}的量
        \item 当我们使用符号$\mathrm d$时,我们默认想要求该符号后附的变量趋近于$0$时的结果
    \end{itemize}
\end{frame}

\begin{frame}{例}
    再次考察函数$f(x)=x^2$,计算其在$x=2$处的导数
\end{frame}

\begin{frame}{投票}
    
\end{frame}

\begin{frame}{Note}
    \begin{itemize}
        \item 探讨一个物体在某个具体时刻是否在运动是没有意义的;
        \item 同样的,探讨一个函数在某个具体的点$x_0$\dots
    \end{itemize}
    因此:
    \begin{itemize}
        \item 使用导数计算出速度为$0$,并不表示物体就是静止的。
        导数是\textbf{某一点附近的最佳直线近似},速度只是近似为$0$.
        \item 扩展到函数性态的研究\dots
    \end{itemize}
\end{frame}

\section{导数的几何理解}

\begin{frame}{例}
    尝试对函数$f(x)=x^2$求导?
\end{frame}

\begin{frame}{例}
    利用几何,求函数$\displaystyle f(x)=\frac1x \ (x>0)$的导数
\end{frame}

\begin{frame}{例}
    利用几何,求函数$\displaystyle f(x)=\sqrt x$的导数
\end{frame}

\begin{frame}{例}
    利用几何,求函数$\displaystyle f(x)=\sin x$的导数
\end{frame}

\section{四则运算,链式法则}

\begin{frame}{例}
    尝试用几何的方法,证明导数的加法与乘法法则
\end{frame}

\begin{frame}{链式法则}
    求函数
    \[
    f(x)=\sin(x^2)
    \]
    的导函数.
\end{frame}

\begin{frame}{链式法则}
    求函数
    \[
    I(x)=f(g(x))
    \]
    的导函数.
\end{frame}


% \section*{}
% \begin{frame}{反馈问卷}
%     \begin{figure}[H]
%         \centering
%         \includegraphics[width=0.5\linewidth]{qrcode.png}
%     \end{figure}
% \end{frame}


% -------------------------------------------------------
% \section*{}
% \begin{frame}
% \vspace{25pt}
% \[
% \text{\Huge Office Hour}
% \]
% \end{frame}

\end{document}
\documentclass[]{beamer}
\usepackage[utf8]{inputenc}
\usepackage{xeCJK}
\usepackage{graphicx}
\usepackage{subfigure}
\usepackage{mathtools}
\usepackage{utopia} %font utopia imported
\usetheme{CambridgeUS}
\usecolortheme{dolphin}
\usefonttheme{professionalfonts}
\usepackage{natbib}
\usepackage{hyperref}
\usepackage{fontspec}
\usepackage{setspace}
\usepackage{float}
\usepackage{extarrows}
% \usepackage{enumitem}

\setCJKmainfont{SourceHanSansSC-Regular.otf}[Path=../, BoldFont=bold.otf]

\setbeamerfont{title}{size=\Large}
\setbeamerfont{subtitle}{size=\small}
\setbeamerfont{date}{size=\small}
\setbeamerfont{institute}{size=\small}

\setstretch{1.3}
% \setlength{\parindent}{2em}
% \setlength{\parskip}{0pt}

% \setlist[itemize]{leftmargin=2em}

% ↓↓↓ Modify this ↓↓↓
\title{高等数学I\quad 习题课12}
\subtitle{函数性态的研究,不定积分}
\date[2025.12.4]{2025.12.4}
% ↑↑↑ Modify this ↑↑↑

\author[上海科技大学]{}
\institute[]{上海科技大学}

\begin{document}

\begin{frame}
    \vspace{15pt}
    \titlepage
\end{frame}

% \begin{frame}{习题课11 反馈}
%     \begin{columns}
%         \begin{column}{0.5\textwidth}
%             \begin{figure}[H]
%                 \centering
%                 \includegraphics[width=1.0\linewidth]{fb1.png}
%                 \caption{课程质量}
%             \end{figure}
%         \end{column}

%         \begin{column}{0.5\textwidth}
%             \begin{figure}[H]
%                 \centering
%                 \includegraphics[width=1.0\linewidth]{fb2.png}
%                 \caption{课堂氛围}
%             \end{figure}
%         \end{column}
%     \end{columns}
% \end{frame}

\begin{frame}{关于后续习题课}
    \begin{itemize}
        \item 根据期中考试成绩反馈,我们会在习题课上尽可能地增加基础运算的练习
        \item 关于几何直观,请大家观看3Blue1Brown的视频:
        \item \url{https://www.bilibili.com/video/BV1qW411N7FU}
        \item 如果时间足够,我们会尽可能地提供几何直观的讲解
    \end{itemize}
\end{frame}


\begin{frame}{目录}
    \vfill
    \tableofcontents[hideallsubsections]
    \vfill
\end{frame}

\AtBeginSection[ ]
{
\begin{frame}{目录}
    \vfill
    \tableofcontents[currentsection,hideallsubsections]
    \vfill
\end{frame}
}

\section{函数性态的研究}

\subsection{单调性}

\begin{frame}{定理4.11,4.12}
    设函数$f(x)\in C[a,b]\cap D(a,b)$,则$f(x)$在$(a,b)$上单调增加的充分必要条件是:
    \[
    \forall x\in(a,b),\quad f'(x)\ge 0
    \]
    若要求$f(x)$在$(a,b)$上严格单调增加,则还要求在$(a,b)$的任何子区间上,$f'(x)$不恒等于$0$.
\end{frame}

\begin{frame}{例4.37}
    求$f(x)=2x^3-9x^2+12x-3$的单调区间.
\end{frame}

\begin{frame}{例4.40}
    证明:当$x>0$时,$\displaystyle x-\frac{x^3}{6}<\sin x<x$.
\end{frame}

\subsection{极值与最值}

\begin{frame}{极值第一判别法}
    设函数$f(x)$在点$x_0$的某个邻域内$U(x_0,\delta)$连续,且在去心邻域内可导,
    \begin{itemize}
        \item [(1)] 若在$x_0$左侧$f'(x_0)<0$,$x_0$右侧$f'(x_0)>0$,则$x_0$是$f(x)$的极小值点;
        \item [(2)] 若在$x_0$左侧$f'(x_0)>0$,$x_0$右侧$f'(x_0)<0$,则$x_0$是$f(x)$的极大值点;
        \item [(3)] 若在$x_0$两侧$f'(x_0)$同号,则$x_0$不是$f(x)$的极值点.
    \end{itemize}
\end{frame}

\begin{frame}{极值第二判别法}
    设函数$f(x)$在点$x_0$有二阶导数,且$f'(x_0)=0$,则
    \begin{itemize}
        \item [(1)] 若$f'(x_0)<0$,则$x_0$是$f(x)$的极大值点;
        \item [(2)] 若$f'(x_0)>0$,则$x_0$是$f(x)$的极小值点;
    \end{itemize}
\end{frame}

\begin{frame}{例4.45}
    \begin{columns}
        \begin{column}{0.6\textwidth}
            设有一块边长为$a$ m的正方形铁皮,在它的四个角上各剪去一个相同边长的小正方形,然后将它沿虚线折起(如图),做成一个无盖的铁盒子. 
            问剪去的小正方形边长$x$为多少米时,能使盒子的容积最大,并求其最大容积.
        \end{column}
        \begin{column}{0.4\textwidth}
            \begin{figure}
                \centering
                \includegraphics[width=\linewidth]{fig4.5.png}
            \end{figure}
        \end{column}
    \end{columns}
\end{frame}

\subsection{凸性}

\begin{frame}{定义}
    设函数$f(x)$在区间$I$连续,若
    \[
    \forall x_1,x_2\in I,\theta\in(0,1),\quad f\Big(\theta x_1+(1-\theta)x_2\Big)\le \theta f(x_1)+(1-\theta)f(x_2),
    \]
    则称函数$f$在该区间上是下凸的(\textit{convex}).

    若将$\le$换为$\ge$,则称$f$在该区间上是上凸的(\textit{concave}).

    若将$\le$换为$<$,则称$f$是严格下凸的.
\end{frame}

\begin{frame}{图解}
    \begin{figure}
        \centering
        \includegraphics[width=1.0\textwidth]{convex.png}
    \end{figure}
\end{frame}

\begin{frame}{应用}
    思考:如果一个函数在整个定义域上都是下凸的,那么这个函数有几个极值点?
\end{frame}

\begin{frame}{第一判别法}
    设函数$f(x)$在区间$(a,b)$内可导,若导函数$f'(x)$在$(a,b)$内严格单调增加(减少),则该函数在$(a,b)$内是严格下凸(上凸)的.
\end{frame}

\begin{frame}{一阶条件}
    \[
    f(x)\ \text{convex}\quad\Leftrightarrow\quad \forall x,y\in I,f(y)\ge f(x)+f'(x)(y-x)
    \]
    \begin{figure}[H]
        \centering
        \includegraphics[width=1.0\textwidth]{convex_1.png}
    \end{figure}
    Image from SI251, Convex Optimization
\end{frame}

\begin{frame}{二阶条件}
    设函数$f(x)$在$(a,b)$内二阶可导,则当$f''(x)>0$时,$f(x)$在$(a,b)$内下凸;当$f''(x)<0$时,$f(x)$在$(a,b)$内上凸.
\end{frame}

\subsection{渐近线}

\begin{frame}{定义}
    若连续曲线$C$上的点$P$沿着曲线无限地远离原点$O$时,点$P$与某一定直线$L$的距离趋于零,即$\lim\limits_{|OP|\rightarrow\infty}\text{dist} (P,L)=0$,则称直线$L$为曲线$C$的渐近线
    \begin{figure}[H]
        \centering
        \includegraphics[width=0.5\textwidth]{jianjinxian.png}
    \end{figure}
\end{frame}

\begin{frame}{分类}
    \begin{itemize}
        \item 铅直渐近线
        \item 水平渐近线
        \item 斜渐近线
    \end{itemize}
\end{frame}

\begin{frame}{铅直渐近线}
    若当$x\rightarrow x_0$(或$x_0^+,x_0^-$)时,$f(x)\rightarrow \infty$,即
    \[
    \lim_{x\rightarrow x_0}f(x)=\infty,
    \]
    则称$x=x_0$是曲线$y=f(x)$的铅直渐近线.
\end{frame}

\begin{frame}{水平渐近线}
   若当$x\rightarrow \infty$(或$+\infty,-\infty$)时,$f(x)\rightarrow b$,即
   \[
   \lim_{x\rightarrow\infty}f(x)=b,
   \] 
   则称直线$y=b$是曲线$y=f(x)$的水平渐近线.
\end{frame}

\begin{frame}{斜渐近线}
    若当$x\rightarrow \infty$(或$+\infty,-\infty$)时,曲线$y=f(x)$与直线$y=ax+b$的距离趋于$0$,即
    \[
    \lim_{x\rightarrow\infty}(f(x)-ax-b)=0,
    \]
    则$y=ax+b$是曲线$y=f(x)$的斜渐近线
\end{frame}

\begin{frame}{例 (24Fall Midterm, 14)}
    全面讨论函数
    \[
    f(x)=(x+2)e^{1/x}
    \]
    的性态(定义域与值域、间断点、零点、单调区间、极值点、上下凸区间、拐点、渐近线)
    
    并作草图.
\end{frame}

\section{不定积分}

\begin{frame}{Motivation}
    先前我们利用导数做了:
    \begin{itemize}
        \item 给定汽车行驶的距离,得到车的“瞬时速度”
    \end{itemize}
    现在我们关心这个问题的逆过程:
    \begin{itemize}
        \item 假设你坐在车里,看不到外界的情况,但是能看到仪表盘
        \item 你要怎样通过已有的信息,得到汽车已经行驶过的距离?
    \end{itemize}
\end{frame}

\begin{frame}{数学原理分析}
    已知速度函数$v(t)$,希望找到距离函数$s(t)$.

    已知关系:
    \[
    v(t)=\frac{\mathrm ds}{\mathrm dt}
    \]
\end{frame}

\begin{frame}{原函数的定义}
    课本P232, 定义5.2

    设函数$f(x)$在区间$I$上有定义,若存在函数$F(x)$使得
    \[
    F'(x)=f(x),
    \]
    则称函数$F(x)$是$f(x)$在$I$上的\textbf{一个}原函数.

    若一个函数有原函数,则其必有无穷多个原函数. 我们称$F(x)+C$($C$为任意常数)为$f(x)$的全体原函数.
\end{frame}

\begin{frame}{不定积分的定义}
    设函数$f(x)$在区间$I$上存在原函数,则称$f(x)$在$I$上的\textbf{全体}原函数为$f(x)$在$I$上的不定积分,记作
    \[
    \int f(x)\mathrm dx
    \]
    其中记号$\int$称为不定积分号,$f(x)$为被积函数,$x$称为积分变量.

    由原函数的定义立即得
    \[
    \int f(x)\mathrm dx = F(x)+C
    \]
    其中$C$为任意常数,也称为不定积分常数.
\end{frame}

\begin{frame}{重要定理 1 (不定积分的导数)}
    若函数$f(x)$在区间$I$上存在原函数,则
    \[
    \Big(\int f(x)\mathrm dx\Big)'=f(x)
    \]
    或,等价地,
    \[
    \mathrm d\Big(\int f(x)\mathrm dx\Big)=f(x)
    \]
\end{frame}

\begin{frame}{重要定理 2 (导数的不定积分)}
    若函数$f(x)$在区间$I$上可导,则
    \[
    \int f'(x)\mathrm dx=f(x)+C,
    \]
    或,等价地,
    \[
    \int \mathrm d f(x)=f(x)+C
    \]
\end{frame}

\begin{frame}{重要定理 3 (线性)}
    若函数$f(x),g(x)$在区间$I$上都存在原函数,$\alpha,\beta\in\mathbb R$且不同时为零,则
    \[
    \int [\alpha f(x)+\beta g(x)]\mathrm dx=\alpha \int f(x)\mathrm dx+\beta\int g(x)\mathrm dx
    \]
\end{frame}

\begin{frame}{基本积分表,课本P241}
    背,或者学会如何推导
    \begin{columns}
        \begin{column}{0.5\textwidth}
            \begin{figure}[H]
                \centering
                \includegraphics[width=0.9\textwidth]{int1.png}
            \end{figure}
        \end{column}
        \begin{column}{0.5\textwidth}
            \begin{figure}[H]
                \centering
                \includegraphics[width=0.9\textwidth]{int2.png}
            \end{figure}
            \begin{figure}[H]
                \centering
                \includegraphics[width=0.9\textwidth]{int3.png}
            \end{figure}
        \end{column}
    \end{columns}
\end{frame}

\begin{frame}{简例}
    求不定积分
    \[
    \int (3\sqrt x-2e^x+\frac{1}{\sqrt x}-\frac{1}{1-x^2})\mathrm dx
    \]
\end{frame}

\begin{frame}{简例}
    求不定积分
    \[
    \int \frac{\mathrm dx}{x^2(x^2+1)}
    \]
\end{frame}

\subsection{换元法}

\begin{frame}{第一换元法:链式法则}
    设函数$F(u)$是$f(u)$在区间$I$上的一个原函数,又$u=\varphi(x)$在区间$I$上可导且其值域为$R(\varphi)\subset I$,则有
    \[
    \int f[\varphi(x)]\varphi'(x)\mathrm dx=F[\varphi(x)]+C.
    \]
\end{frame}

\begin{frame}{例}
    求
    \[
    \int \frac{1}{\arctan x(1+x^2)}\mathrm dx
    \]
\end{frame}

\begin{frame}{第二换元法:反函数求导}
    设函数$x=\varphi(t)$在区间$I$上可导且导数$\varphi'(t)\ne 0$,且$\int f[\varphi(t)]\varphi'(t)\mathrm dt=F(t)+C$,则
    \[
    \int f(x)\mathrm dx = F[\varphi^{-1}(x)]+C
    \]
    不好理解?
\end{frame}

\begin{frame}{例}
    求
    \[
    \int \frac{\mathrm dx}{\sqrt{x^2+a^2}}\mathrm dx \quad(a>0)
    \]
\end{frame}

\begin{frame}{例}
    求
    \[
    \int \sqrt{a^2-x^2}\mathrm dx \quad(a>0)
    \]
\end{frame}

\begin{frame}{例}
    求
    \[
    \int \frac{\mathrm dx}{x\sqrt{x^2+1}}
    \]
\end{frame}

\begin{frame}{分部积分法:导数的乘法法则}
    设函数$u(x),v(x)$在区间$I$上可导,且$\int u'(x)v(x)\mathrm dx$存在,则
    \[
    \int u(x)v'(x)\mathrm dx=u(x)v(x) - \int u'(x)v(x)\mathrm dx
    \]
    用于$u(x)v'(x)$不易积分,但$v'(x)u(x)$易求出的情况
\end{frame}

\begin{frame}{例}
    求
    \[
    \int xe^{-x}\mathrm dx
    \]
\end{frame}

\begin{frame}{例}
    求
    \[
    \int \arcsin x\mathrm dx
    \]
\end{frame}

\begin{frame}{例}
    求
    \[
    \int e^{ax}\sin bx\mathrm dx\quad (a\ne 0)
    \]
\end{frame}




% -------------------------------------------------------
% \section*{}
% \begin{frame}{反馈问卷}
%     \begin{figure}[H]
%         \centering
%         \includegraphics[width=0.5\linewidth]{qrcode.png}
%     \end{figure}
% \end{frame}
% \begin{frame}
% \vspace{25pt}
% \[
% \text{\Huge }
% \]
% \end{frame}

\end{document}